\documentclass[a4paper,11pt,twocolumn,twoside]{article}
\usepackage[dvips]{graphicx}
\usepackage{sepln_en}
\usepackage{fullname}
\usepackage[utf8]{inputenc}
\usepackage[spanish,es-nosectiondot, es-tabla, es-noindentfirst, es-nolists]{babel}

\input epsf

\setlength\titlebox{4in} %esto por defecto

\title{Título completo del artículo centrado con Times New Roman negrita, tamaño 16}

\author {\textbf{Nombre Apellidos1,$^1$} \textbf{Nombre Apellidos2$^2$}\\
$^1$Universidad o lugar de trabajo\\
$^2$Universidad o lugar de trabajo\\
Información de contacto\\
}

\seplntranstitle{Título traducido del artículo centrado con Times New Roman negrita, cursiva, tamaño 14}

\seplnclave{Palabras, palabras, palabras en castellano...}

\seplnresumen{Resumen del artículo en castellano con una sangría a izquierda y
derecha de 1 cm, justificado por ambos lados, con tamaño de fuente
11.}


\seplnkey{Palabras, palabras, palabras en inglés...}

\seplnabstract{Resumen del artículo en inglés con una sangría a izquierda y
derecha de 1 cm, justificado por ambos lados, con tamaño de fuente
11.}

\firstpageno{1}

\begin{document}

% la siguiente instrucción sólo se debe usar si el abstract sobrescribe el texto
% la longitud variará según se necesite

%\setlength\titlebox{20cm} % se aumenta el tamaño del espacio reservado para datos de título

\label{firstpage} \maketitle

%\begin{abstract}
%Resumen del artículo con una sangría a izquierda y derecha de 0.32
%cm, justificado por ambos lados, con tamaño de fuente 11.
%
%\end{abstract}

\section{Título de nivel uno}

\cite{piad2019general}

\section*{Agradecimientos}

\bibliographystyle{fullname}
\bibliography{bibliography}

\end{document}
