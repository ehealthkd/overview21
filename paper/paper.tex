\documentclass[a4paper,11pt,twocolumn,twoside]{article}
\usepackage[dvips]{graphicx}
\usepackage{sepln_en}
\usepackage{fullname}
\usepackage[utf8]{inputenc}
% \usepackage[spanish,es-nosectiondot, es-tabla, es-noindentfirst, es-nolists]{babel}

\input epsf

\setlength\titlebox{4in} %esto por defecto

\title{Overview of the eHealth Knowledge Discovery Challenge at IberLEF 2021}

\author {\textbf{Nombre Apellidos1,$^1$} \textbf{Nombre Apellidos2$^2$}\\
$^1$Universidad o lugar de trabajo\\
$^2$Universidad o lugar de trabajo\\
Información de contacto\\
}

\seplntranstitle{Resumen de la Tarea de Descubrimiento de Conocimiento en Salud en IberLEF 2021}

\seplnclave{Palabras, palabras, palabras en castellano...}

\seplnresumen{Resumen del artículo en castellano con una sangría a izquierda y
derecha de 1 cm, justificado por ambos lados, con tamaño de fuente
11.}


\seplnkey{Palabras, palabras, palabras en inglés...}

\seplnabstract{Resumen del artículo en inglés con una sangría a izquierda y
derecha de 1 cm, justificado por ambos lados, con tamaño de fuente
11.}

\firstpageno{1}

\begin{document}

% la siguiente instrucción sólo se debe usar si el abstract sobrescribe el texto
% la longitud variará según se necesite

%\setlength\titlebox{20cm} % se aumenta el tamaño del espacio reservado para datos de título

\label{firstpage} \maketitle

%\begin{abstract}
%Resumen del artículo con una sangría a izquierda y derecha de 0.32
%cm, justificado por ambos lados, con tamaño de fuente 11.
%
%\end{abstract}

\section{Introduction}


The accelerated growth of the Internet and the increased production of textual resources in all areas of human endeavour has created both new opportunities and new challenges for the research community.
On one hand, larger datasets can be collected and used to build increasingly powerful machine learning models~(e.g., GPT-3 and related).
On the other hand, is becoming increasingly difficult to organise, categorise, cross-reference, and fact-check the textual information available online.
The ease of access to both publication and consumption of textual information is arguably one of the root causes of the increasingly worrying phenomenon of fake news.

Staying up-to-date on information about topics of interest is crucial in technical or scientific domains.
For this purpose, specialists rely on a combination of curated soruces (e.g., domain-specific repositories like arxiv, bioarxiv, semantic scholar) and technologies for search and recommendation (e.g., google scholar).
These resources significantly improve the experience of collecting and consuming large amounts of relevant information on a specific topic.
Tools like Connected Papers~\cite{} and Papers with Code~\cite{} are one step beyond the indexing of documents, providing summarised and structured representations of the content in a collection of documents.
However, it remains an open problem to automatically combine, summarise, and present the relevant information in a collection of documents in a semantic structure~(e.g., a knowledge graph) that allows a specialist to quickly grasp the essential concepts of a specific knowledge field.

A potential solution to this problem would require methods to automatically detect in natural language the most relevant concepts and the factual statements in which they are related, possibly normalise them into well-established taxonomies and classifications, and store them in computational data structures where they can linked with related concepts, e.g., knowledge graphs.
The first step of this process, i.e., the detection of relevant concepts and semantic relations in text, is already a challenging computational task, given the complexity and variability of natural language.
To encourage research and development in this area, several academic competitions have been organised through the years by organizations such as CLEF, SEMEVAL, and more recently, IBERLEF.

In this context, the eHealth Knowledge Discovery Challenge is designed to foster the development of automatic knowledge discovery systems for natural language sentences in a cross-domain and multi-lingual setting.
Concretely, a token-level annotation model of 4 entities and 13 semantic relations is defined, and a corpus of more than 1500 sentences from different factual sources~(i.e., Spanish Medline articles, Spanish Wikinews articles, and English biomedical preprints related to COVID-19) is annotated.
Two computational tasks are defined: the detection of multi-span, non-contiguous, and potentially overlapping named entities; and the detection of semantic relations between them.
A shared annotation campaign was organized, where a total of 9 participants presented 10 different systems with varied levels of performance.

The eHealth-KD task focuses on cross-domain, multi-lingual and low-resource solutions. This setting presents a significant challenge to existing state-of-the-art methods, which often rely on large amounts of training data.
A succesful approach to the eHealth-KD must leverage transfer learning across different domains and languages, since the vast majority of the training examples are provided in Spanish language and from the Medline domain, while only a small development set is available in the remaining settings.
With this added complexity, we expect to encourage solutions that can be deployed in low-resource environments, where is unfeasible to train large language models over longs periods of time.

The remaining of this paper is organized as follows.
Section~\ref{sec:task} describes the eHealth-KD tasks in greater detail, including the annotation model and performance metrics.
Section~\ref{sec:resources} describes the corpora and other resources created for this challenge and presents some qualitative analysis of their characteristics.
Section~\ref{sec:systems} describes the different systems presented in the challenge and summarises the main approaches and most common characteristics they share.
Section~\ref{sec:results} presents the main results of the challenge.
Section~\ref{sec:discussion} discusses the most interesting insights of the challenge and highlights both the most relevant lessons learned and potential improvements for future similar endeavours.

\section{Tasks Descriptions}\label{sec:task}

The eHealth-KD challenge consists on the automatic sentence-level annotation
of multi-token entities and binary relations among them.

A custom annotation model has been defined, comprising 4 entity types and 13 relations,
that attempts to capture a large part of factual semantics in technical documents,
including enciclopedias, news, and scientific papers.

The annotation model is explained in detail in \namecite{piad2019general}.

A brief explanation is presented next.

-- Explicar las entidades y relaciones como siempre.

The challenge has been divided into two different subtasks:
entity recognition and relation extraction.

To evaluate each subtask, we count correct, partial, missing, and spurious
annotations, and weight them according to an $F_1$ measure, independently per task.

-- Fórmulas

Furthermore, the challenge is graded in three different scenarios:
one for each subtask and one where both subtasks are performed in sequence.
This is the main scenario, which ultimately decides the winner of the challenge,
although all other results are interesting.

\section{Corpora and Resources}\label{sec:resources}

The Challenge is based on a corpus of documents composed of
sentences taken from previous challenges and other resources,
and newly annotated sentences.

Three different sources of textual information are sampled: medline articles,
wikinews articles, and biomedical papers from the CORD-19 corpus.

The medline and wikinews articles are in Spanish, while CORD are in English.

From the 2019 and 2020 challenges, X sentences where taken.

For this new edition, Y sentences were annotated by a team of annotators.
The team composition is 4 junior annotators and 3 senior annotators.
2 junior annotators annotated each sentence, and a senior annotator performed the merging.

The newly annotated sentences are used solely for evaluation purposes,
while all training and development resources are reused from previous challenges.

Overall, the composition of the corpus is presented in Table~\ref{tab:corpus}.

\begin{table}
  \caption{Composition of the corpus, highlighting resources from previous
  challenges and newly annotated sentences.\label{tab:corpus}}
\end{table}

\section{Systems Descriptions}\label{sec:systems}
´
The challenge caught the attention of 8 participants from across the globe,
who presented a variety of approaches clustered around deep learning architectures.

%-- Describir los sistemas
\begin{description}
  \item[uhKD4:]  system is built upon two independent deep-learning-based architectures. Accordingly, two
  different models are defined and each task is carried out separately. Task A is approached as a
  sequence labelling problem in which each token from an input sequence is assigned a label that
  represents the combination of the BILUOV entity tagging scheme with each one of the possible
  types of an entity. The BILUOV tags correspond to: Begin, to represent the start of an entity;
  Inner, to represent its continuation; Last, to represent its end; Unit, to represent single word
  entities; Other, to represent words that are not a part of any entity; and oVerlapping, to represent
  words that belong to multiple entities.Thus, the output of the model considers 21 different labels: the
  O label and the combination of the remaining tags (BILUV) and the entity types (Concept,
  Action, Predicate and Reference). The proposed approach to Task B is to solve a multi-class
  classification problem, in which given a sentence and a highlighted pair of entities, one of the
  predefined relations is assigned to occur from the first entity toward the second one. A new
  artificial relation class none is defined to symbolize the non-occurrence of any relation between
  a pair of entities.

  \item[Vicomtech Joint Classifier:] This model is the result of the revision of our participation in the previous eHealth-KD edition
  [2]. Most of the changes introduced in the current edition aim at simplifying components of
  the architecture that seemed redundant or needlessly complex. In addition, we have forgone
  the ability to predict multi-word discontinuous and/or overlapping entities like the shown in
  Figure 1, which in any case required elaborate post-processing to yield acceptable results.
  It consists of a single end-to-end deep neural network with
  pre-trained BERT models as the core for the semantic representation of the input texts, that predicts all
  the output variables—entities and relations—at the same time, modelling the whole problem jointly.
  The main change w.r.t. the original implementation affects the representation of relations. The Joint
  Classifier model achieved the first position in the main scenario of the competition and ranked second
  in the rest of the scenarios.

  \item[Vicomtech  Seq2Seq:] uses an approach based on an
  encoder-decoder model. It transduces the input text into an output sequence by reading the input text.
  The target sequence is a compact representation of the information contained in the gold-labels of
  the datasets. This approach showed a promising performance despite not being competitive enough.
  However, it poses an interesting potential future work.
  This paradigm is way more flexible because it can ingest a sequence of any length,
  and output another arbitrary sequence. That is, the output sequence is not tied to the structure
  of the input; a text-to-text model can potentially encode any sort of information

  \item[PUCRJ-PUCPR-UFMG:] a multilingual BERT-based system for joint entity recognition and relation extraction
  in multidomain texts. Our end-to-end multitasking model benefits from the transformer architecture,
  which has proved to capture better the global dependencies of the input text. Also, the use of a multilin-
  gual model contributed to our system to perform well even in the set of tests containing non-Spanish
  sentences. The method consists in an end-to-end multilingual
  BERT-based system that jointly predicts both entities and relations. During training, the
  proposed multi-task system is fine-tuned in 3 sequential steps: the first one prioritizes the entity
  recognition task whereas the second gives precedence to the relation extraction one. Finally,
  the third and last step is trained for both tasks using a multi-task strategy.

  \item[UH-MMM:]  solution solves both tasks separately and sequentially. Thus, independent models
  with different architectures and features were trained to solve the NER and RE problems. The
  main distinction between the two architectures raises from the type of problem they solve. The
  first task is posed as a tag prediction problem that takes the raw text of a sentence as input and
  outputs two independent tag sequences: one in the BILOUV tag system for entity prediction and
  another with the tags corresponding to each entity type (Concept, Action, Reference, Predicate).
  The BILOUV tag scheme classification corresponds to Begin, for the start of an entity; Inner, for
  the token in the middle; Last for the ending token; Unit, to represent single token entities; Other
  to represent tokens that do not belong to any entity, and the oVerlapping tag is used to deal
  with tokens that belong to multiple entities. On the other hand, the second task is addressed as
  a series of pairwise queries among the entities present in the target sentence, oriented towards
  identifying the relevant relations between the previously extracted entities.

  \item[(IXA):]  use the pre-trained Language Model (LM), namely XML-
  RoBERTa-base, to provide a Cross-lingual representation of tokens and the ability to transfer learning
  from general domains. system was
  designed as a pipeline of generic sequence labellers, each of them independently fine-tuned for each
  subtask. The generic sequence labeller consists of a feed-forward network that learns how to align
  a sequence of tokens into a sequence of labels regardless of the language and domain. This simple
  straightforward system ranked in the third position in the main and Entity recognition scenario and
  widely outperformed the other systems in the relation extraction scenario.

  \item[Maoqin:]  Several deep learning models are used in plain text documents, such as
  BERT-CRF, BiLSTM-CRF. Only the best result running in the local has been submitted. But the final
  result of my method is indeed very low, with a score of 0.173 (F1), ranking 9th on the leaderboard.
  Additional work needs to be done to improve our final result and complete the subtask B: relation
  extraction. For one method, the pre-trained BERT model has been taken as the basic representation.
  For another method, BiLSTM has been preferred. The system achieved unsatisfied results in the
  challenge for official evaluation in subtask A. Due to the colab’s time and resources limitation,
  some strategies were not achieved. And subtask B: Relation Extraction has not been completed.

  \item [GuanZhengyi:] . We mainly used the classic BERT+BiLSTMs+CRF model architecture and replaced BERT
  with BETO as the pre-training model. BETO is a BERT model pre-training on Spanish text. And we
  add CNN before BiLSTM for further feature extraction. Our model architecture performed well in the
  training set and development set for identifying the types of entities
  uses the pre-training model BETO with stronger text feature
  representation capabilities as the feature representation layer. After reading the eHealth 2020
  papers, we fine-tuned the above architecture. We added CNN to further extract features of
  text. We regard NER as a classification problem at the token level. We combined BETO with
  CNN, BiLSTM and CRF model to extract global and local features of the text. That means we
  input these word vectors into CNN, BiLSTM and CRF to calculate the corresponding scores and
  perform entity recognition and BIO labeling.

  \item[JAD:] The model consist of identifying key-phrases
  and relations among them using a pre-defined system. It was a simple model that summarizes
  some parts of a general approach to NLP problem. The system is easy to train and test using
  cloud services like Google Colab. It did not perform very well at the competition.
  Our system is based on a pre-trained multi-lingual BERT[2] layer which pre-processes the input sentence and then feeds the generated
embeddings to the deep learning model. 
\end{description}

\section{Challenge Results}\label{sec:results}

Table~\ref{tab:results} summarizes the main results in the challenge.

\begin{table}
  \caption{Results.\label{tab:results}}
\end{table}

In Scenario 1, the best performing system is\dots

-- Describir los demás escenarios

\section{Discussion}\label{sec:discussion}

-- Ver los sistemas con sus features y qué features son mejores o peores en cada tarea

-- Comparar con los baselines en cada tarea, hasta qué nivel está resuelta cada tarea

-- Hablar brevemente de la evolución de los sistemas en las 4 ediciones,
cómo se han ido moviendo hacia transformers

-- Hablar de la modelación de las entidades y relaciones como tareas separadas,
la desventaja que puede representar eso porque los sistemas que han ganado hacen
ambas cosas a la vez, o sea hay representación conjunta que es válida

-- Hablar de las relaciones como problema binario, que no tiene en cuenta que ciertas
relaciones implican posiblemente la aparición de otras.

-- Hablar de que el orden en que el humano anota no es el mismo que un sistema,
y que si se pudiera aprender algo de esto, dar con el dataset no solo las anotaciones
sino en qué orden fueron realizadas.

-- Hablar de la necesidad de cambiar de tareas hacia algo que refleje el valor
del conocimiento extraído, como question answering.

\section{Conclusions}\label{sec:conclusions}

\section*{Acknowledgments}

\bibliographystyle{fullname}
\bibliography{bibliography}

\end{document}
